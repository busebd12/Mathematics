\documentclass[executivepaper]{article}

\usepackage{mathtools}

\everymath{\displaystyle}

\usepackage{amssymb}

\usepackage{amsfonts}

\usepackage{kantlipsum,graphicx}

\usepackage{amsmath}

\usepackage[utf8]{inputenc}

\usepackage{sectsty}

\usepackage{float}

\usepackage{commath}

\usepackage{amsmath}

\usepackage{amsthm}

\usepackage{adjustbox}

\usepackage{fancyhdr}
 
\pagestyle{fancy}

\fancyhf{}

\rhead{Brendan Busey}

\lhead{Homework 9}

\rfoot{Page \thepage}

\renewcommand{\headrulewidth}{1pt}

\renewcommand{\footrulewidth}{1pt}

\begin{document}

\begin{flushleft}

1) W know that any matrix in $O(2)$ must either take the form of

\begin{center}

$\bigl(\begin{smallmatrix}
a&-b \\ b&a
\end{smallmatrix} \bigr)$

\end{center}

or

\begin{center}

$\bigl(\begin{smallmatrix}
a&b \\ b&-a
\end{smallmatrix} \bigr)$

\end{center}

And, since any matrix in $O(2)$ either reflects or rotates a vector in $\mathbb{R}^2$, the determinant of any matrix in $O(2)$ will be $\pm 1$. Therefore, the only elements that commute are $Z(G)=\{ \pm I \}$, or the identity matrix. 

\end{flushleft}

\begin{flushleft}

2)

\begin{proof}

Doing the prime-factorization of 2016, we get $2016=2^5 \cdot 3^2 \cdot 7$. So, by the \textit{Fundamental Theorem of Finitely Generated Abelian Groups}, we have

\begin{center}

$\mathbb{Z}_{2} \times \mathbb{Z}_{2} \times \mathbb{Z}_{2} \times \mathbb{Z}_{2} \times \mathbb{Z}_{2} \times \mathbb{Z}_{3} \times \mathbb{Z}_{3} \times \mathbb{Z}_{7}$

\vspace{2mm}

$\mathbb{Z}_{2} \times \mathbb{Z}_{2} \times \mathbb{Z}_{2} \times \mathbb{Z}_{2} \times \mathbb{Z}_{2} \times \mathbb{Z}_{9} \times \mathbb{Z}_{7}$

\vspace{2mm}

$\mathbb{Z}_{2} \times \mathbb{Z}_{16} \times \mathbb{Z}_{3} \times \mathbb{Z}_{3} \times \mathbb{Z}_{7}$

\vspace{2mm}

$\mathbb{Z}_{2} \times \mathbb{Z}_{16} \times \mathbb{Z}_{9} \times \mathbb{Z}_{7}$

\vspace{2mm}

$\mathbb{Z}_{2} \times \mathbb{Z}_{2} \times \mathbb{Z}_{8} \times \mathbb{Z}_{3} \times \mathbb{Z}_{3} \times \mathbb{Z}_{7}$

\vspace{2mm}

$\mathbb{Z}_{2} \times \mathbb{Z}_{2} \times \mathbb{Z}_{8} \times \mathbb{Z}_{9} \times \mathbb{Z}_{7}$

\vspace{2mm}

$\mathbb{Z}_{2} \times \mathbb{Z}_{2} \times \mathbb{Z}_{2} \times \mathbb{Z}_{4} \times \mathbb{Z}_{3} \times \mathbb{Z}_{3} \times \mathbb{Z}_{7}$

\vspace{2mm}

$\mathbb{Z}_{2} \times \mathbb{Z}_{2} x \mathbb{Z}_{2} \times \mathbb{Z}_{4} \times \mathbb{Z}_{9} \times \mathbb{Z}_{7}$

\vspace{2mm}

$\mathbb{Z}_{32} \times \mathbb{Z}_{3} \times \mathbb{Z}_{3} \times \mathbb{Z}_{7}$

\vspace{2mm}

$\mathbb{Z}_{32} \times \mathbb{Z}_{9} \times \mathbb{Z}_{7}$

\end{center} 

\end{proof}

\end{flushleft}

\begin{flushleft}

3)

\end{flushleft}

\begin{flushleft}

4)

\begin{proof}

Let the order of $G$ be $m$ = $p_1^{\alpha_1} \ldots p_k^{\alpha_k}$. It is known that $G$ is a direct product of $p$-groups, say:

\begin{center}
 
$G = G_1 \times \ldots \times G_k$

\end{center}

where each $G_i$ is a $p_i$-group. By the \textit{fundamental theorem of finite abelian groups}, each $G_i$ is isomorphic to a direct product of cyclic groups of the form

\begin{center}

$\mathbb{Z}_{{p_i}^{\beta_1}} \times \ldots \times \mathbb{Z}_{{p_i}^{\beta_l}},$

where $\beta_1, \ldots, \beta_l$ are positive integers such that $\sum_{j=1}^l \beta_j = \alpha_i$.

\end{center}

Now if $n$ divides $m$, then we must have

\begin{center}

$n = p_1^{\gamma_1} \ldots p_k^{\gamma_k}$

for some $\gamma_1, \ldots, \gamma_k$ with $0 \leq \gamma_i \leq \alpha_i$.

\end{center}

Which brings us to the following claim: \textit{Each $G_i$ has a subgroup of order $p_i^{\gamma_i}$}

Proof: As above, we have that

$ G_i \cong \mathbb{Z}_{{p_i}^{\beta_1}} \times \ldots \times \mathbb{Z}_{{p_i}^{\beta_l}} $ where $\beta_1, \ldots, \beta_l$ are positive integers such that $\sum_{j=1}^l \beta_j = \alpha_i$. 
Now since $0 \leq \gamma_i \leq \alpha_i$, we can find$l$ numbers $\delta_1, \ldots , \delta_l$ such that $\gamma_i = \sum_{j=1}^l \delta_j$, and $0 \leq \delta_j \leq \beta_j$. (This choice of numbers is not necessarily unique).
Then, $p_i^{\delta_j} | p_i^{\beta_j}$ for each $j = 1, \ldots , l$. Hence, for each factor $\mathbb{Z}_{{p_i}^{\beta_j}}$, there exists a subgroup of order $p_i^{\delta_j}$, namely $\mathbb{Z}_{{p_i}^{\delta_j}}$ (using the fact that the converse of Lagrange's theorem is true for finite cyclic groups). Taking the direct product of each of these subgroups, we get a new subgroup $G_i'$ of $G_i$:

\begin{center}

$G_i' \cong \mathbb{Z}_{{p_i}^{\delta_1}} \times \ldots \times \mathbb{Z}_{{p_i}^{\delta_l}}$

\end{center}

The order of this subgroup is $p_i^{\delta_1} \times \ldots \times p_i^{\delta_l} = p_i^{\delta_1 + \ldots + \delta_l} = p_i^{\gamma_i} $. So, we have found a subgroup of $G_i$ of order $p_i^{\gamma_i}$, as required. So, each factor $G_i$ in the product $G = G_1 \times \ldots \times G_k$ has a subgroup $G_i'$ of order $p_i^{\gamma_i}$. Therefore, $G$ has a subgroup

\begin{center}

$G_1' \times G_2' \times \ldots \times G_k'$

of order $p_1^{\gamma_i}...p_k^{\gamma_k} = n$,

\end{center}

which completes the proof.

\end{proof}

\end{flushleft}

\begin{flushleft}

5) 
\begin{proof}

Similar to problem $2$, we first compute the prime factorization of 16

\begin{center}

$16=2^4$

\end{center}

Then, by the \textit{Fundamental Theorem of Finitely Generated Abelian Groups}, we have

\begin{center}

$G \cong \mathbb{Z}_{2} \times \mathbb{Z}_{2} \times \mathbb{Z}_{2} \times \mathbb{Z}_{2} \times \mathbb{Z}_{2}$

\vspace{2mm}

$G \cong \mathbb{Z}_{2} \times \mathbb{Z}_{2} \times \mathbb{Z}_{4}$

\vspace{2mm}

$G \cong \mathbb{Z}_{2} \times \mathbb{Z}_{8}$

\vspace{2mm}

$G \cong \mathbb{Z}_{4} \times \mathbb{Z}_{4}$

\vspace{2mm}

$G \cong \mathbb{Z}_{16}$

\end{center}

Now, if we can rule out $\mathbb{Z}_{2} \times \mathbb{Z}_{2} \times \mathbb{Z}_{4}$, $\mathbb{Z}_{2} \times \mathbb{Z}_{8}$, and $\mathbb{Z}_{16}$, as $\mathbb{Z}_{4}$, $\mathbb{Z}_{8}$, and $\mathbb{Z}_{16}$ each have elements of order 2, but $a^2=b^2$ in each case. However, in $\mathbb{Z}_{4} \times \mathbb{Z}_{4}$, the order of $(1,0)$ is 4 and $(1,0)^2=(2,0)$, and for $(0,1)$, the order is also 4. But, $(0,1)^2=(0,2) \neq (2,0)$. Therefore, the isomorphism class of $G$ is $\mathbb{Z}_{4} \times \mathbb{Z}_{4}$. 

\end{proof}

\end{flushleft}

\end{document}