\documentclass[executivepaper]{article}

\usepackage{mathtools}

\everymath{\displaystyle}

\usepackage{amssymb}

\usepackage{commath}

\usepackage{kantlipsum,graphicx}

\usepackage{amsmath}

\usepackage{pgfplots}

\usepackage[utf8]{inputenc}

\usepackage{sectsty}

\usepackage{float}

\usepackage[utf8]{inputenc}

\usepackage[english]{babel}

\usepackage{amsthm}

\usepackage{fancyhdr}

\pagestyle{fancy}

\fancyhf{}

\rhead{Final Exam}

\lhead{Brendan Busey}

\rfoot{Page \thepage}

\sectionfont{\large}

\subsectionfont{\normalsize}

\renewcommand{\headrulewidth}{1.5pt}

\renewcommand{\footrulewidth}{1.5pt}

\newtheorem{lemma}{Lemma}

\begin{document}

\begin{flushleft}

A2)

\begin{proof}

Let $a, b \in G$. Since $G$ is a group, we know that it contains the respective inverses $a^{-1}, b^{-1}$. Then, we have

\begin{center}

$\phi \left(ab \cdot a^{-1}b^{-1}\right)=\phi \left(ab\right) \circ \phi \left(a^{-1}b^{-1}\right)$

\vspace{2mm}

$\phi \left(ab \cdot a^{-1}b^{-1}\right)=\phi \left(a \right) \circ \phi \left(b \right) \circ \phi \left(a^{-1} \right) \circ \phi \left(b^{-1} \right)$

\vspace{2mm}

$\phi \left(ab \cdot a^{-1}b^{-1}\right)=\phi \left(a \right) \circ \phi \left(b \right) \circ \left[\phi \left(a^{-1} \right)\right]^{-1} \circ \left[\phi \left(b^{-1} \right)\right]^{-1}$

\vspace{2mm}

$\phi \left(ab \cdot a^{-1}b^{-1}\right)=\left[\phi \left(a^{-1} \right)\right]^{-1} \circ \left[\phi \left(b^{-1} \right)\right]^{-1} \circ \phi \left(a \right) \circ \phi \left(b \right)$

\vspace{2mm}

$\phi \left(ab \cdot a^{-1}b^{-1}\right)=a^{-1}b^{-1}ab$

\end{center}

\end{proof}

\end{flushleft}

\vspace{5mm}

\begin{flushleft}

A3)

Non-empty:

\begin{center}

$(1)^{2} + (1)^{2}=c^{2}$

\vspace{2mm}

$1 + 1=c^{2}$

\vspace{2mm}

$2=c^{2}$

\vspace{2mm}

$c=\sqrt{2}$

\vspace{2mm}

and we can write $\sqrt{2}$ as $\frac{\sqrt{2}}{1} \in \mathbb{Q}^{*}$

\end{center}

$ab^{-1} \in H$ part:

\begin{center}

Let $(a+bi)$ and $(c+di)^{-1}$ be in $H$. Indeed, we have

\begin{center}

$(a+bi)(c+di)^{-1}=\left(\frac{(a+bi) \cdot (c-di)}{(c-di) \cdot (c-di)}\right)=\left(\frac{ac+bd}{c^2+d^2}\right) + \left(\frac{bc-ad}{c^2+d^2}\right)i \in H$

\end{center}

\end{center}

\end{flushleft}

\pagebreak

\begin{flushleft}

A5)

\begin{proof}

Given elements $a$, $b$ of $G$, and $ab$ has finite order n. Hence, $\abs{ab}=n$ $\iff$ $(ab)^{n}=e$. We need to show that $n$ is the smallest integer such that $(ab)^{n}=e$.

\begin{center}

$(ab)^{n}=e \implies b(ab) \ldots (ab)a=bea \implies (ba)^{n}=e \implies \abs{ba} \leq n$.

\end{center}

Now, show that there is no positive integer $m$ such that $m < n$ and $(ba)^{m}=e$. But, if $\abs{ba} < n$, then we could apply the same reasoning to find that $\abs{ab} \leq \abs{ba} < \abs{ab}$, which is absurd. So, $\abs{ba}=\abs{ab}=n$.

\end{proof}

\end{flushleft}

\begin{flushleft}

B1)

Before we begin, we will prove some lemmas that will be useful later.

\vspace{2mm}

\begin{lemma}

Let $\left(R, +, \cdot \right)$ be a ring whose zero is $0_{R}$. Then, $\forall ~ x \in R: 0_{R} \cdot x=0_{R}=x \cdot 0_{R}$. In other words, the zero is a zero element for the ring product, thereby justifying its name.

\end{lemma}

\begin{proof}

Because $\left(R, +, \cdot \right)$ is a ring, $\left(R, +\right)$ is a group. Since $0_{R}$ is the identity in $\left(R, +\right)$, we have $0_{R} + 0_{R}=0_{R}$. From the cancellation laws, all group elements are cancelable, so every 	element of $\left(R, +\right)$ is cancelable for $+$. Hence,

\begin{center}

$x \cdot \left(0_{R} + 0_{R}\right)=x \cdot 0_{R}$

\vspace{2mm}

$\implies \left(x \cdot 0_{R}\right) + \left(x \cdot 0_{R}\right)=x \cdot 0_{R}$

\vspace{2mm}

$\implies \left(x \cdot 0_{R}\right) + \left(x \cdot 0_{R}\right)=\left(x \cdot 0_{R}\right) + 0_{R}$

\vspace{2mm}

$\implies x \cdot 0_{R}=0_{R}$

\end{center}

Then,

\begin{center}

$\left(0_{R} + 0_{R}\right) \cdot x=0_{R} \cdot x$

\vspace{2mm}

$\implies \left(x \cdot 0_{R}\right) + \left(x \cdot 0_{R}\right)=0_{R} \cdot x$

\vspace{2mm}

$\implies \left(x \cdot 0_{R}\right) + \left(x \cdot 0_{R}\right)=0_{R} + \left(0_{R} \cdot x\right)$

\vspace{2mm}

$\implies 0_{R} \cdot x=0_{R}$

\end{center}

\end{proof}

\vspace{3mm}

\begin{lemma}

Let $\left(R, +, \cdot \right)$ be a ring. Suppose further that $R$ is not the null ring. Let $f \in R$ such that $f^{k}=0$ with $k \geq 1$ implies $f=0$. Then, $f$ is a zero divisor.

\end{lemma}

\begin{proof}

Let $0_{R}$ be the zero fo $R$. By hypothesis, there exists $n \geq 1$ such that $x^{n}=0_{R}$. If $n=1$, then $x=0_{R}$. By hypothesis, $R$ is not the null ring, so we may choose $y \in R ~ \backslash ~ \{0\}$. Now, by our lemma 2, we have

\begin{center}

$y \cdot x=y \cdot 0_{R}=0_{R}$

\end{center}

Therefore, $x$ is a zero divisor in $R$. If $ n \geq 2$, define $y=x^{n-1}$. Then, we have

\begin{center}

$y \cdot x=x^{n-1} \cdot x=x^{n}=0_{R}$

\end{center}

So, $x$ is the zero divisor in $R$.

\end{proof}

Now that we have finished with that, onto the main event!

\vspace{2mm}

\begin{lemma}

Let $\left(R, +, \cdot\right)$ be an integral domain. Then, $R$ is reduced.

\end{lemma}

\begin{proof}

Let $x \in R$ such that $x^{k}=0$ with $k \geq 1$ implies $x=0$. Then, by our lemma 2, $x$ is a zero divisor. So, by the definition of an integral domain, this means that $x=0$. Therefore, the only element $x \in R$ such that $x^{k}=0$ with $k \geq 1$ implies $x=0$ of $R$ is 0. Thus, $R$ is reduced.

\end{proof}

Finally, an example of a reduced ring that is not an integral domain would be $\mathbb{Z}\left[x,y\right] ~ \backslash ~ (xy)$ or $\mathbb{Z} \times \mathbb{Z}$

\end{flushleft}

\vspace{5mm}

\begin{flushleft}

B2) A complete characterization of the set of left ideals of the ring $R$ of $2 \times 2$ matrices over $\mathbb{R}$ would be all of the matrices of the form

\begin{center}

$\begin{pmatrix}

ar_{1} & br_{1}\\

ar_{2} & br_{2}

\end{pmatrix}$ 

\end{center}

where $r_{1}$ and $r_{2}$ run over all the real numbers, i.e. all of the matrices whose rows are scalar multiples of vector $(a,b)$. 

\end{flushleft}

\begin{flushleft}

B3)

\begin{proof}

Associativity: Let $u_{1}, u_{2}, u_{3} \in U(R)$. Then, in particular, $u_{1}, u_{2}, u_{3}$ are in $R$ and since multiplication in $R$ is associative, it is associative in $U(R)$

\vspace{2mm}

Invertibility: Let $u_{1} \in R$. Then, there exists a $u_{1}^{-1} \in R$. Therefore, $u_{1}u_{1}^{-1}=e=u_{1}^{-1}u_{1}$, where $e$ is the identity element of $R$. Hence, $u_{1}=\left(u_{1}^{-1}\right)^{-1}$. Thus, $u_{1}^{-1} \in U(R)$ whenever $u_{1} \in R$

\vspace{2mm}

Identity: $R$ has a unit identity. Let this unity be denoted by $e$ (from above). Then, $e^{-1}e=e=ee^{-1}$ if $e^{-1}$ exists. But, $ee=e$ and therefore, $e=e^{-1}$ and therefore $e \in U(R)$.

\vspace{2mm}

To show that it is closed under the binary operation:

\begin{center}

If $a,b \in U(R)$, then $a^{-1}, b^{-1}$ exist in $R$. Therefore, $(ab)\left(b^{-1}a^{-1}\right)=a\left(bb^{-1}\right)a^{-1}=aea^{-1}=aa^{-1}=e$ and $R$ is commutative, so the same holds on the left. Therefore, $U(R)$ is closed $\left(a, b \in U(R) ~ implies ~ that ~ ab \in U(R)\right)$. Thus, $U(R)$ is a group. 

\end{center}

\vspace{3mm}

A necessary and sufficient condition for the matrix

\begin{center}

$\begin{pmatrix}

a & b\\

c & d

\end{pmatrix}$ 

\end{center}

to be a unit in the ring Mat$_{2 \times 2}\left(\mathbb{Z}\right)$ is that the determinant must be $\pm 1$. In other words, the units of the ring Mat$_{2 \times 2}\left(\mathbb{Z}\right)$ is the set of $2 \times 2$ matrices with determinant equal to $\pm 1$.

\end{proof}

\end{flushleft}

\end{document}