\documentclass[executivepaper]{extarticle}
\usepackage[margin=3.0cm]{geometry}
\usepackage{fancyhdr}
\usepackage{extsizes}
\usepackage{enumerate}
\usepackage{enumitem}
\usepackage{mathtools}

\begin{document}

\vspace*{-40mm}

\begin{center}

\textbf{Modern Class Notes}

\end{center}

\section*{1/12/16}

\vspace{-5.5mm}

\noindent \rule{2cm}{0.5pt}

A group is an object that tries to describe the symmetry of something \\

Going to study:

\begin{enumerate}

\item Permutation groups

\item Matrix groups

\item Cosets and Legrange's Theorem

\item And other constructs

\end{enumerate}

Review of some discrete math stuff: \\

\begin{enumerate}

\item A function $f: x \rightarrow y$ (where X and Y are sets) is an assignment of an element of Y for each element of X

\item Fact: let $f: x \rightarrow y$ be a function. Then:

\begin{enumerate}

\item $f$ is invertible $\iff f$ is one to one

\end{enumerate}

\end{enumerate}

Ex)Let x={1, 2, 3,...,n}\\

$\pi(1)=2$

$\pi(2)=3$

$\pi(n-1)=n$

$\pi(n)=1$\\

A permutation is a one-to-one and onto function from a finite set to itself \\

At least 3 ways of recording a permutation: \\

\begin{enumerate}

\item Two line notation

\item One line notation (works for permutations of order up to 9)

\item Cycle notation: start with a number and write down where it goes and then repeat this process for the rest of the numbers

\item 

\end{enumerate}


\end{document}