\documentclass[executivepaper]{article}

\usepackage{mathtools}

\everymath{\displaystyle}

\usepackage{kantlipsum,graphicx}

\usepackage{amssymb}

\usepackage{commath}

\usepackage{amsmath}

\begin{document}

\vspace*{-40mm}

\begin{center}

Homework 2

\end{center}

\begin{flushright}

Brendan Busey

\end{flushright}

\begin{flushleft}

1) Let the differential equation $u''(x)=0$ for $-1 \leq x \leq 1$ subject to the following boundary conditions 

\begin{center}

$-u'(-1)=\lambda u(-1)$ and $u'(-1)=\lambda u(1)$

\end{center}

be given.

\vspace{3mm}

Then, we have the following: 

\begin{center}

$u''(x)=0$

\vspace{3mm}

$u'(x)=c$, for some constant $c \in \mathbb{Z}$

\vspace{3mm}

$u(x)=c_{1}(x)+c_{0}$, for some constants $c_{0}, c_{1} \in \mathbb{Z}$, which is our general solution.

\end{center}

So, our eigenfunctions are $u(x)=x$ and $u(x)=1$. Then, in order to use the initial conditions, we have to calculate the following:

\begin{center}

$u(1)=1$ \quad \quad $u(-1)=1$ \quad \quad $u'(x)=1$

\end{center}

Now, to solve for our $\lambda$ values, we have to use the initial conditions:

\vspace{3mm}

For $u(x)=x$, we have:

\begin{center}

$u'(1)=\lambda u(1)$  \quad \quad \quad $-u'(-1)=\lambda u(-1)$

$1=\lambda (1)$ \quad \quad \quad $-u'(-1)=(1)(-1)$

$1=\lambda$ \quad \quad \quad $-u'(-1)=-1$

\quad \quad \quad $u'(-1)=1$

\end{center}

For $u(x)=1$, we have:

\begin{center}

$u'(1)=\lambda u(1)$  \quad \quad \quad $-u'(-1)=\lambda u(-1)$

$u'(1)=\lambda(1)$  \quad \quad \quad $-u'(-1)=(1)(-1)$

$1=\lambda (1)$ \quad \quad \quad $-u'(-1)=-1$

$\lambda=1$ \quad \quad \quad $-u'(-1)=-1$

\end{center}

Thus, our eigenfunction(s) are $u(x)=x$ and $u(x)=1$ and our corresponding eigendata is $\lambda=1$.

\end{flushleft}

\begin{flushleft}

2) For the $S-eigenvalue$ problem, the eigenfunctions that were obtained do satisfy the given boundary conditions. However, the functions obtained in the \textit{Dirchlet-Laplacian} and the \textit{Neumann-Laplacian} did not satisfy the respective boundary conditions. Also, the eigenfunctions obtained in this eigenvalue problem are consistent with the eigenfunctions in the other two eigenvalue problems.

\end{flushleft}

\pagebreak

\vspace*{-40mm}

\begin{flushleft}

3) In order to show that our two eigenfunctions are in $L^2(-1,1)$, we have to evaluate the following integral for each function:

\begin{center}

$\int_{-1}^{1} \abs{u(x)}^2 dx$

\end{center}

For $u(x)=x$, we have:

\begin{center}

$\int_{-1}^{1} \abs{u(x)}^2 dx$

\vspace{1mm}

$\implies \int_{-1}^{1} \abs{x}^2 dx$

\vspace{1mm}

$\implies \frac{1}{3}x^3 \Big|_{-1}^{1}$

\vspace{1mm}

$\implies \bigg[\frac{1}{3}(1)^3\bigg]- \bigg[\frac{1}{3}(-1)^3\bigg]$

\vspace{1mm}

$\implies \frac{1}{3}-\bigg(-\frac{1}{3}\bigg)=\frac{2}{3}$

\end{center}

\vspace{3mm}

For $u(x)=1$, we have:

\begin{center}

$\int_{-1}^{1} \abs{u(x)}^2 dx$

\vspace{1mm}

$\implies \int_{-1}^{1} \abs{1}^2 dx$

\vspace{1mm}

$\implies \int_{-1}^{1} 1 dx$

\vspace{1mm}

$\implies x \Big|_{-1}^{1}$

\vspace{1mm}

$\implies \bigg[ 1 \bigg]- \bigg[-1\bigg]=2$

\end{center}

So, from the above work, we can see that only $u(x)=x \in L^2(-1,1)$.

\end{flushleft}

\begin{flushleft}

4) To see if $u(x)=x$ and $u(x)=1$ are orthogonal, we have to integrate their product from $-1$ to $1$:

\begin{center}

$\int_{-1}^{1} (1)(x) dx$

\vspace{1mm}

$\implies \int_{-1}^{1} x dx$

\vspace{1mm}

$\implies \frac{1}{2}x^2 \Big|_{-1}^{1}$

\vspace{1mm}

$\implies \bigg[\frac{1}{2}(1)^2\bigg]- \bigg[\frac{1}{2}(-1)2\bigg]$

\vspace{1mm}

$\implies \frac{1}{2}-\bigg(-\frac{1}{2}\bigg)=0$

\end{center}

\pagebreak

\vspace*{-40mm}

So, the $S-eigenfunctions$ are orthogonal to each other since their inner (dot) product is zero. Geometrically, when we consider these functions as ``vectors," this means that the ``vectors" are perpendicular to each other.

\end{flushleft}

\begin{flushleft}

5) All of the functions from classes one through four are harmonic functions since their respective $\Delta u(x,y)=0$

\end{flushleft}

\begin{flushleft}

6) In order to show that the functions in class-1 are in $L^2(\Omega)$, we have to evaluate the following integral for each function:

\begin{center}

$\int_{-1}^{1} \int_{-1}^{1}\abs{u(x,y)}^2 \,dx dy$

\end{center}

\vspace{2mm}

For $S_{0}(x,y)=1$:

\vspace{3mm}

If we consider the outer integral, we have:

\begin{center}

$\int_{-1}^{1} \int_{-1}^{1}\abs{1}^2 \,dx dy$

\vspace{1mm}

$\implies \int_{-1}^{1} \int_{-1}^{1} 1 \,dx dy$

\vspace{1mm}

$\implies x \Big|_{-1}^{1}$

\vspace{1mm}

$\implies \bigg[1\bigg]-\bigg[-1\bigg]=2$

\end{center}

Now, if we consider the integral, we have:

\begin{center}

$\int_{-1}^{1} 2  \ dy$

\vspace{1mm}

$\implies 2y \Big|_{-1}^{1}$

\vspace{1mm}

$\implies \bigg[2(1)\bigg]-\bigg[2(-1)\bigg]=2(-2)=4$

\end{center}

\vspace{3mm}

For $S_{1,1}(x,y)=\cosh(vx)\cos(vy)$, we have:

\begin{center}

$\int_{-1}^{1} \int_{-1}^{1} \abs{\cosh(vx)\cos(vy)}^2 \,dx dy$

\vspace{1mm}

$\implies \int_{-1}^{1} \int_{-1}^{1} \cosh^2(vx)\cos^2(vy) \,dx dy$

\end{center}

Now, if we consider the inner integral, we have:

\begin{center}

$\int_{-1}^{1} \cosh^2(vx) \ dx$

\vspace{1mm}

$\implies v \int_{-1}^{1} \cosh^2(vx) \ dx$

\vspace{1mm}

$\implies v \int_{-1}^{1} \frac{1}{2} (1+\cosh(2vx) \ dx$

\vspace{1mm}

$\implies v \bigg[\frac{1}{2} \bigg(x+\frac{1}{2}\sinh(2vx)\bigg) \Big|_{-1}^{1}\bigg]$

\pagebreak

\vspace*{-40mm}

$\implies v \bigg[\frac{1}{2} \bigg(x+\frac{1}{2}\sinh(vx)\cosh(vx)\bigg) \Big|_{-1}^{1}\bigg]$

\vspace{3mm}

$v \bigg[\frac{1}{2} \bigg(1+\frac{1}{2}\sinh(v(1))\cosh(v(1))\bigg) \bigg]-v \bigg[\frac{1}{2} \bigg(-1+\frac{1}{2}\sinh(v(-1))\cosh(v(-1))\bigg) \bigg]$

\vspace{3mm}

$\implies v \bigg[\frac{1}{2} \bigg(1+\frac{1}{2}\sinh(v)\cosh(v)\bigg) \bigg]-v \bigg[\frac{1}{2} \bigg(-1+\frac{1}{2}\sinh(-v)\cosh(-v)\bigg) \bigg]$

\end{center}

Now, if we consider the outer integral, we have:

\begin{center}

$\int_{-1}^{1} \cos^2(vy) \ dy$

\end{center}

Now, the astute reader may recall the following (trig) identity:

\begin{center}

$cos^2(x)=\frac{1}{2}+\cos(2x)$

\end{center}

Using this identity, our integral becomes:

\begin{center}

$\int_{-1}^{1} \frac{1+\cos(v \cdot vy)}{2} \ dy$

\vspace{1mm}

$\implies \frac{1}{2} \int_{-1}^{1} 1+\cos(v \cdot vy) \ dy$

\vspace{1mm}

$\implies \frac{1}{2} \bigg(y \Big|_{-1}^{1} + \frac{\sin(2vy)}{4v} \Big|_{-1}^{1} \bigg)$

\vspace{1mm}

$\implies \frac{1}{2} \bigg(2 + \bigg(\frac{\sin(2v)}{4v} - \frac{\sin(-2v)}{4v}\bigg) \bigg)$

\vspace{1mm}

$\implies \frac{1}{2} \bigg(2 + 2\frac{\sin(2v)}{4v} \bigg)$

\vspace{1mm}

$\implies \bigg(1 + \frac{\sin(2v)}{4v} \bigg)$

\end{center}

So, our final result is:

\begin{center}

$v \bigg[\frac{1}{2} \bigg(1+\frac{1}{2}\sinh(v)\cosh(v)\bigg) \bigg]-v \bigg[\frac{1}{2} \bigg(-1+\frac{1}{2}\sinh(-v)\cosh(-v)\bigg) \bigg] \cdot \bigg(1 + \frac{\sin(2v)}{4v} \bigg)$

\end{center}

For $S_{1,2}(x,y)=\cos(vx)\cosh(vy)$, we have:

\begin{center}

$\int_{-1}^{1} \int_{-1}^{1} \abs{\cos(vx)\cosh(vy)}^2 \,dx dy$

\vspace{1mm}

$\implies \int_{-1}^{1} \int_{-1}^{1} \cosh^2(vx)\cos^2(vy) \,dx dy$

\end{center}

Looking at the inner integral, we have:

\begin{center}

$\int_{-1}^{1} \cos^2(vx) \ dx$

\end{center}

and by a similar argument from the previous calculation in this question, our solution is:

\pagebreak

\vspace*{-40mm}

\begin{center}

$1 + \frac{\sin(2v)}{4v}$

\end{center}

Looking at the outer integral, we have:

\begin{center}

$\int_{-1}^{1} \cosh^2(vy) \ dx$

\end{center}

and by a similar argument from the previous calculation in this question, our solution is:

\begin{center}

$v \bigg[\frac{1}{2} \bigg(1+\frac{1}{2}\sinh(v)\cosh(v)\bigg) \bigg]-v \bigg[\frac{1}{2} \bigg(-1+\frac{1}{2}\sinh(-v)\cosh(-v)\bigg) \bigg]$

\end{center}

So, our final answer is

\begin{center}

$v \bigg[\frac{1}{2} \bigg(1+\frac{1}{2}\sinh(v)\cosh(v)\bigg) \bigg]-v \bigg[\frac{1}{2} \bigg(-1+\frac{1}{2}\sinh(-v)\cosh(-v)\bigg) \bigg] \cdot 1 + \frac{\sin(2v)}{4v}$

\end{center}

\end{flushleft}

\begin{flushleft}

7) Yes, I do think that the functions in classes two, three, and four are also in $L^2(\Omega)$ because they all share similar vector calculus properties such at the laplacian, gradient, etc.

\end{flushleft}

\begin{flushleft}

8) For $S_{1,2}(x,y)=\cos(vx)\cosh(vy)$, we have:

\begin{center}

$\int_{-1}^{1} \int_{-1}^{1} \cos(vx)\cosh(vy) \ dx dy$

\end{center}

Considering the inner integral, we have:

\begin{center}

$\int_{-1}^{1} \cos(vx) \ dx dy$

\vspace{1mm}

$\implies \frac{\sin(vx)}{v} \Big|_{-1}^{1}$

\vspace{1mm}

$\implies \frac{\sin(v)}{v} - \bigg[\frac{\sin(-v)}{v}\bigg]=2\frac{\sin(v)}{v}$

\end{center}

Considering the outer integral, we have:

\begin{center}

$\int_{-1}^{1} \cosh(vy) \ dy$

\vspace{1mm}

$\implies \frac{\sinh(vy)}{v} \Big|_{-1}^{1}$

\vspace{1mm}

$\implies \frac{\sinh(v)}{v} - \bigg[\frac{\sinh(-v)}{v}\bigg]=2\frac{\sinh(v)}{v}$

\end{center}

So, our final answer is

\begin{center}

$2\frac{\sin(v)}{v} \cdot 2\frac{\sinh(v)}{v}=2\bigg(\frac{\sin(v)}{v} \cdot \frac{\sinh(v)}{v}\bigg)$

\end{center}

\pagebreak

\vspace*{-40mm}

For $S_{2,1}(x,y)=\sinh(vx)\sin(vy)$, we have:

\begin{center}

$\int_{-1}^{1} \int_{-1}^{1} \sinh(vx)\sin(vy) \ dx dy$

\end{center}

Considering the inner integral, we have:

\begin{center}

$\int_{-1}^{1} \sinh(vx) \ dx$

\vspace{1mm}

$\implies \frac{\cosh(vx)}{v} \Big|_{-1}^{1}$

\vspace{1mm}

$\implies \frac{\cosh(v)}{v} - \bigg[\frac{\cosh(-v)}{v}\bigg]$

\vspace{1mm}

$\implies \frac{\cosh(v)}{v} - \bigg[\frac{\cosh(-v)}{v}\bigg]=0$

\end{center}

Considering the outer integral, we have:

\begin{center}

$\int_{-1}^{1} \sin(vy) \ dy$

\vspace{1mm}

$\implies -\frac{\cos(vy)}{v} \Big|_{-1}^{1}$

\vspace{1mm}

$\implies -\frac{\cos(v)}{v} - \bigg[-\frac{\cos(-v)}{v}\bigg]$

\vspace{1mm}

$\implies -\frac{\cosh(v)}{v} + \bigg[\frac{\cos(-v)}{v}\bigg]=$

\vspace{1mm}

$\implies -\frac{\cos(v)}{v} + \bigg[\frac{\cos(v)}{v}\bigg]=0$

\end{center}

So, our final answer is $0$.

\end{flushleft}

\begin{flushleft}

9)According to our calculations, we are able to find that all the functions from the four classes are orthogonal except for $S_{1,2}(x,y)=\cos(vx)\cosh(vy)$.

\end{flushleft}

\begin{flushleft}

10) After looking over the harmonic functions for the one and two dimensional case(s), I noticed that the one-dimensional cases are either constant or linear functions, while the second-dimensional case(s) are periodic functions. However, both the one-dimensional and the two-dimensional case(s) have their laplacian equal to zero, which I was not expecting considering that the dimensions are different. 

\end{flushleft}

\end{document}