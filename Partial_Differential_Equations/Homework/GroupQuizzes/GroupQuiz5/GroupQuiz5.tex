\documentclass[executivepaper]{article}

\usepackage{mathtools}

\everymath{\displaystyle}

\usepackage{amssymb}

\usepackage{amsfonts}

\usepackage{kantlipsum,graphicx}

\usepackage{amsmath}

\usepackage[utf8]{inputenc}

\usepackage{sectsty}

\usepackage{float}

\usepackage{commath}

\usepackage{amsmath}

\usepackage{amsthm}

\usepackage{adjustbox}

\usepackage{fancyhdr}
 
\pagestyle{fancy}

\fancyhf{}

\rhead{Madison Hoff and Brendan Busey}

\lhead{Group Quiz 5}

\rfoot{Page \thepage}

\renewcommand{\headrulewidth}{1pt}

\renewcommand{\footrulewidth}{1pt}

%\newcommand{\norm}[1]{\left\lVert#1\right\rVert}

\begin{document}

\begin{flushleft}

1) Let the differentiable equation $f''(x)+\lambda f(x)=0$ be given and let $g(x)$ be a twice differentiable function. \\ 

\begin{proof}

Now, multiply $f''(x)+\lambda f(x)=0$ by $g(x)$ to obtain

\begin{center}

$f''(x)g(x)+\lambda f(x)g(x)=0$

\end{center}

Now, we want to integrate $f''(x)g(x)+\lambda f(x)g(x)=0$ by parts. But, before we do, let use the \textit{properties of integrals} to obtain

\begin{center}

$\int_{0}^{L} f''(x)g(x) \ dx + \int_{0}^{L} \lambda f(x)g(x) \ dx=0$

\end{center}

Okay, first, consider the case where $\lambda > 0$. Next, integrate $\int_{0}^{L} f''(x)g(x) \ dx$ by parts

\begin{center}

$u=g(x)$ $\quad$ $du=g''(x)$ $\quad$ $dv=f''(x)$ $\quad$ $v=\int f''(x)=f'(x)+c$

\vspace{2mm}

$\int_{0}^{L} f''(x)g(x)= \bigg[g(x)f'(x)\bigg] \bigg|_{0}^{L} - \int_{0}^{L} f'(x)g'(x) \ dx$

\end{center}

Substituting in for $\int_{0}^{L} f''(x)g(x) \ dx$, we have

\begin{center}

$\int_{0}^{L} f''(x)g(x) \ dx + \int_{0}^{L} \lambda f(x)g(x)=0$

\vspace{2mm}

$=\bigg[g(x)f'(x)\bigg]_{0}^{L} - \int_{0}^{L} f'(x)g'(x) \ dx + \int_{0}^{L} \lambda f(x)g(x) \ dx=0$

\end{center}

Now, if we use $f_{n,N}'(x)=-\sqrt{\lambda_{n,D}} f_{n,D}(x)$, and $g_{n,M}'(x)=-\sqrt{\lambda_{m,D}} f_{m,D}(x)$, we get

\begin{center}

$\bigg[g(x) (-\sqrt{\lambda_{n,D}} f_{n,D}(x))\bigg]_{0}^{L} - \int_{0}^{L} -\sqrt{\lambda_{n,D}} f_{n,D}(x) \cdot -\sqrt{\lambda_{m,D}} f_{m,D}(x) \ dx + \int_{0}^{L} \lambda f(x)g(x) \ dx=0$

\end{center}

Then, making the substitution $f(x)=f_{n,N}(x)$ and $g(x)=f_{m,N}(x)$, we have

\begin{center}

$\bigg[f_{m,N}(x) (-\sqrt{\lambda_{n,D}} f_{n,D}(x))\bigg]_{0}^{L} - \int_{0}^{L} -\sqrt{\lambda_{n,D}} f_{n,D}(x) \cdot -\sqrt{\lambda_{m,D}} f_{m,D}(x) \ dx + \int_{0}^{L} \lambda f_{n,N}(x)f_{m,N}(x) \ dx=0$

\end{center}

Simplifying, we have

\begin{center}

$\bigg[f_{m,N}(x) (-\sqrt{\lambda_{n,D}} f_{n,D}(x))\bigg]_{0}^{L} -(\sqrt{\lambda_{n,D}} \cdot \sqrt{\lambda_{m,D}}) \int_{0}^{L} f_{n,D}(x) \cdot  f_{m,D}(x) \ dx + \int_{0}^{L} \lambda f_{n,N}(x)f_{m,N}(x) \ dx=0$

\end{center}

Since we know that the \textit{Dirichlet-Laplacian} eigendata inner-product is zero, the

\begin{center}

$(\sqrt{\lambda_{n,D}} \cdot \sqrt{\lambda_{m,D}}) \int_{0}^{L} f_{n,D}(x) \cdot  f_{m,D}(x) \ dx$

\end{center}

term will be zero, leaving us with

\begin{center}

$\bigg[f_{m,N}(x) (-\sqrt{\lambda_{n,D}} f_{n,D}(x))\bigg]_{0}^{L} + \int_{0}^{L} \lambda f_{n,N}(x)f_{m,N}(x) \ dx=0$

\end{center}

Now, if you evaluate $-\sqrt{\lambda_{n,D}} f_{n,D}(x)$ from $0$ to $L$, we know that by the properties of \textit{Dirichlet-Laplacian} eigendata, this term will go to zero. This leaves us with

\begin{center}

$\int_{0}^{L} \lambda f_{n,N}(x)f_{m,N}(x) \ dx=0$

\vspace{2mm}

$=\lambda \int_{0}^{L} f_{n,N}(x)f_{m,N}(x) \ dx=0$

\vspace{2mm}

$=\int_{0}^{L} f_{n,N}(x)f_{m,N}(x) \ dx=0$

\end{center}

as desired.

\vspace{5mm}

Now, we turn our attention to calculating the \textit{norm} for the \textit{Neumann-Laplacian} eigenfunction(s). First, let the differentiable equation $f''(x)+\lambda f(x)=0$ be given and let $g(x)$ be a twice differentiable function. Now, multiply $f''(x)+\lambda f(x)=0$ by $g(x)$ to obtain

\begin{center}

$f''(x)g(x)+\lambda f(x)g(x)=0$

\end{center}

Now, we want to integrate $f''(x)g(x)+\lambda f(x)g(x)=0$ by parts. But, before we do, let's use the properties of integrals to obtain

\begin{center}

$\int_{0}^{L} f''(x)g(x) \ dx +\int_{0}^{L} \lambda f(x)g(x) \ dx=0$

\end{center}

Next, we integrate $\int_{0}^{L} f''(x)g(x) \ dx$ by parts

\begin{center}

$u=g(x) \quad du=g'(x) \quad dv=f''(x) \quad v=\int f''(x)=f'(x)+c$

\end{center}

We can now re-write $\int_{0}^{L} f''(x)g(x) \ dx$ as

\begin{center}

$\int_{0}^{L} f''(x)g(x) \ dx=\bigg[g(x)f'(x)\bigg]_{0}^{L} - \int_{0}^{L} f'(x)g'(x) \ dx$

\end{center}

If we then substitute in for  $\int_{0}^{L} f''(x)g(x) \ dx$, we have

\begin{center}

$\int_{0}^{L} f''(x)g(x) \ dx + \int_{0}^{L} \lambda f(x)g(x) \ dx=0$

\vspace{2mm}

$=\bigg[g(x)f'(x)\bigg]_{0}^{L} - \int_{0}^{L} f'(x)g'(x) \ dx + \int_{0}^{L} \lambda f(x)g(x) \ dx=0$

\end{center}

Now, if we use $f'_{n, N}(x)=-\sqrt{\lambda_{n, D}} f_{n, D}$ and $g'_{n, N}(x)=-\sqrt{\lambda_{n, D}} f_{n, D}$, we have

\begin{center}

$\bigg[g(x) (\sqrt{\lambda_{n, D}} f_{n, D}(x))\bigg]_{0}^{L} - \int_{0}^{L} - \sqrt{\lambda_{n, D}} f_{n, D}(x) \cdot \sqrt{\lambda_{n, D}} \ dx + \int_{0}^{L} \lambda f(x)g(x) \ dx=0$

\end{center}

If we evaluate $\sqrt{\lambda_{n, D}} f_{n, D}(x)$ from $0$ to $L$, we know that by the properties of \textit{Dirichlet-Laplacian} eigendata, this term will go to zero, leaving us with

\begin{center}

$- \int_{0}^{L} - \sqrt{\lambda_{n, D}} f_{n, D}(x) \cdot \sqrt{\lambda_{n, D}} \ dx + \int_{0}^{L} \lambda f(x)g(x) \ dx=0$

\end{center}

Then, if we factor our constants, we have

\begin{center}

$-(-\sqrt{\lambda_{n, D}})(-\sqrt{\lambda_{n, D}}) \int_{0}^{L} f_{n,D}(x) f_{n, D}(x) \ dx + \int_{0}^{L} \lambda f(x)g(x) \ dx=0$

\end{center}

Then, if we combine our integrals, we have

\begin{center}

$-(-\sqrt{\lambda_{n, D}})(-\sqrt{\lambda_{n, D}}) \int_{0}^{L} f_{n,D}(x) f_{n, D}(x) +  \int_{0}^{L} \lambda f(x)g(x) \ dx=0$

\end{center}

Dividing out our constants, we find

\begin{center}

$\int_{0}^{L} f_{n,D}(x) f_{n, D}(x) \ dx +  \int_{0}^{L} \lambda f(x)g(x) \ dx=0$

\end{center}

Re-expanding our integrals, we have

\begin{center}

$-\int_{0}^{L} f_{n,D}(x) f_{n, D}(x) \ dx + \int_{0}^{L} \lambda f(x)g(x) \ dx=0$

\vspace{2mm}

$\implies \int_{0}^{L} \lambda f(x)g(x) \ dx=\int_{0}^{L} f_{n,D}(x) f_{n, D}(x) \ dx$

\vspace{2mm}

$\implies \int_{0}^{L} \lambda f(x)g(x) \ dx=\sqrt{\frac{L}{2}},  ~ since ~ we ~ know ~ that ~ \norm{f_{n,D}}=\sqrt{\frac{L}{2}} $

\end{center}

Finally, making the substitution $f(x)=f){n, N}(x)$ and $g(x)=f_{n, N}(x)$, we have

\begin{center}

$\int_{0}^{L} \lambda f(x)g(x) \ dx=\sqrt{\frac{L}{2}}$, as desired.

\end{center}

\end{proof}

\end{flushleft}

\begin{flushleft}

2) 
\begin{proof}

First, we begin by supposing that the solution $f(x)$ has the specific form $f(x)=e^{kx}$, so $f''(x)=k^2e^{kx}$. We then plug-in $f(x)$ and $f''(x)$ into our ODE giving us

\begin{center}

$k^2e^{kx}+\lambda e^{kx}=0$

\end{center}

Next, we factor our the $e^{kx}$ term, leaving us with

\begin{center}

$e^{kx}\bigg[k^2+\lambda \bigg]=0$

\end{center}

Which simplifies to

\begin{center}

$k^2+\lambda=0$, which is our \textit{auxiliary equation}

\end{center}

Now, we want to consider the case where $\lambda < 0$, or when $\lambda$ is strictly negative. Using the auxiliary equation $k^2+\lambda=0$, we solve for $k$

\begin{center}

$k=\pm \sqrt{-\lambda}$, when $\lambda < 0$

\end{center}

This gives us the following solutions

\begin{center}

$v_{1}(x)=e^{\sqrt{-\lambda} x}$ $\quad$ and $\quad$ $v_{2}(x)=e^{-\sqrt{-\lambda} x}$

\end{center}

We save showing linear independence for later and instead, impose our \textit{mixed-boundary} conditions

\begin{center}

$v_{1}(x)=e^{\sqrt{-\lambda} x}$ $\quad$ and $\quad$ $v_{2}(x)=e^{-\sqrt{-\lambda} x}$

\vspace{2mm}

$v_{1}(0)=e^{\sqrt{-\lambda} (0)}$ $\quad$ and $\quad$ $v_{2}(x)=e^{-\sqrt{-\lambda} (0)}$

\vspace{2mm}

$v_{1}(0)=e^{0}$ $\quad$ and $\quad$ $v_{2}(x)=e^{0}$

\vspace{2mm}

$v_{1}(0)=1 \neq 0$ $\quad$ and $\quad$ $v_{2}(x)=1 \neq 0$

\end{center}

Therefore, neither $v_{1}(x)$ nor $v_{2}(x)$ are eigenfunctions because they do not satisfy the \textit{mixed-boundary} conditions

\vspace{5mm}

Moving on, we consider the case where $\lambda=0$. When $\lambda=0$, our ODE becomes $f''(x)=0$. We can solve for $f(x)$ by integrating twice

\begin{center}

$f'(x)=c_{1}$

$f''(x)=c_{1}x+c_{2}$

\end{center}

This gives us the following two solutions

\begin{center}

$h_{1}(x)=x$ $\quad$ and $\quad$ $h_{2}(x)=1$

\end{center}

Now, we impose our \textit{mixed-boundary} conditions

\begin{center}

$h_{1}(x)=x$ $\quad$ and $\quad$ $h_{2}(x)=1$

\vspace{2mm}

$h_{1}(0)=0$ $\quad$ and $\quad$ $h_{2}(0)=1 \neq 0$

\vspace{2mm}

$h_{1}(1)=1 \neq 0$

\end{center}

Therefore, neither $h_{1}(x)$ or $h_{2}(x)$ are eigenfunctions because they don't satisfy our \textit{mixed-boundary} conditions

\vspace{5mm}

Finally, we consider the case where $\lambda > 0$, or $\lambda$ is strictly positive. Using the previously calculated equation, $k^2+\lambda=0$, we solve for $k$

\begin{center}

$k=\pm \sqrt{-\lambda}$ $\quad$ or $\quad$ $k=\pm i\sqrt{\lambda}$

\end{center}

So, we obtain the solutions

\begin{center}

$u_{1}(x)=e^{i\sqrt{\lambda} x}$ $\quad$ and $\quad$ $u_{2}(x)=e^{-i\sqrt{\lambda} x}$

\end{center}

However, we only want real valued solutions. So, we use \textit{Euler's formula}

\begin{center}

$\tilde{u}_{1}(x)=\frac{1}{2} \bigg[u_{1}(x) + u_{2}(x)\bigg]$

\vspace{2mm}

$\tilde{u}_{1}(x)=\frac{1}{2} \bigg[\cos\bigg(\sqrt{\lambda} x\bigg) + i\sin\bigg(\sqrt{\lambda x}\bigg) + \cos\bigg(\sqrt{-\lambda x}\bigg) + i\sin\bigg(\sqrt{-\lambda x}\bigg)\bigg]$

\vspace{2mm}

$\tilde{u}_{1}(x)=\cos\bigg(\sqrt{\lambda x}\bigg)$

\vspace{2mm}

$\tilde{u}_{2}(x)=\frac{1}{2i} \bigg[\cos\bigg(\sqrt{\lambda} x\bigg) + i\sin\bigg(\sqrt{\lambda x}\bigg) - \cos\bigg(\sqrt{-\lambda x}\bigg) + i\sin\bigg(\sqrt{-\lambda x}\bigg)\bigg]$

\vspace{2mm}

$\tilde{u}_{2}(x)=\sin\bigg(\sqrt{\lambda x}\bigg)$

\end{center}

Now, we want to impose our \textit{mixed-boundary} conditions

\begin{center}

$\tilde{u}_{1}(x)=\cos\bigg(\sqrt{\lambda x}\bigg)$ $\quad$ $\tilde{u}_{2}(x)=\sin\bigg(\sqrt{\lambda x}\bigg)$

\vspace{2mm}

$\tilde{u}_{1}(0)=1 \neq 0$ $\quad$ $\tilde{u}_{2}(x)=\sin(0)=0$

\end{center}

Because  $\tilde{u}_{2}(x)=\sin\bigg(\sqrt{\lambda x}\bigg)$ satisfies our first boundary condition, we want to make it so that $\tilde{u}_{2}'(L)=0$. So

\begin{center}

$\sqrt{\lambda} \cos\bigg(\sqrt{\lambda} L\bigg)=0$

\end{center}

Solving for $\lambda$, we have

\begin{center}

$\sqrt{\lambda} \cos\bigg(\sqrt{\lambda} L\bigg)=0$

\vspace{2mm}

$\cos\bigg(\sqrt{\lambda} L\bigg)=0$

\vspace{2mm}

$\cos^{-1}\Bigg(\cos\bigg(\sqrt{\lambda L}\bigg)\Bigg)=\cos^{-1}(0)$

\end{center}

So, $(\lambda L)=(2n-1)\frac{\pi}{2}$, where $2n-1$ are odd numbers. So, $\lambda=\bigg(\frac{(n \pi - \frac{\pi}{2})}{L}\bigg)^2,  ~ \forall ~ n \in \mathbb{N}$

Therefore, our \textit{DN-Laplacian} eigenvalues are denoted as:

\begin{center}

$\lambda_{n, DN}=\bigg(\frac{(n \pi - \frac{\pi}{2})}{L}\bigg)^2, ~ \forall ~ n \in \mathbb{N}$

\end{center}

Thus, our \textit{DN-Laplacian} eigenfunctions are

\begin{center}

$f_{n,DN}(x)=\sin\bigg(\frac{(n \pi - \frac{\pi}{2})}{L} x\bigg), ~ \forall ~ n \in \mathbb{N}$

\end{center}

\end{proof}

\end{flushleft}

\begin{flushleft}

3)

\begin{proof}

First, we suppose that the solutions are of the form $y(x)=e^{kx}$. Now, we have already demonstrated the derivation of our \textit{auxiliary equation} in the previous question, so we will not repeat that work again here. So, using our previously obtained \textit{auxiliary equation}

\begin{center}

$k^2+\lambda=0$

\end{center}

and the work shown in the previous question, we know that our two solutions are

\begin{center}

$y_{1}(x)=e^{\sqrt{-\lambda} x}$ $\quad$ and $\quad$ $y_{2}(x)=e^{-\sqrt{-\lambda} x}$

\end{center}

Now, to apply our \textit{mixed-boundary conditions}, we take some derivatives

\begin{center}

$y_{1}'(x)=\sqrt{-\lambda} e^{\sqrt{-\lambda} x}$ $\quad$ and $\quad$ $y_{2}'(x)=-\sqrt{-\lambda} e^{\sqrt{-\lambda} x}$

\end{center}

Applying our \textit{mixed-boundary conditions}, we have

\begin{center}

$y_{1}'(x)=\sqrt{-\lambda} e^{\sqrt{-\lambda} x}$ $\quad$ and $\quad$ $y_{2}'(x)=-\sqrt{-\lambda} e^{\sqrt{-\lambda} x}$

\vspace{2mm}

$y_{1}'(0)=\sqrt{-\lambda} e^{\sqrt{-\lambda} (0)}$ $\quad$ and $\quad$ $y_{2}'(0)=-\sqrt{-\lambda} e^{\sqrt{-\lambda} (0)}$

\vspace{2mm}

$0=\sqrt{-\lambda}$ $\quad$ and $\quad$ $0=-\sqrt{-\lambda}$

\end{center}

Therefore, neither $y_{1}(x)$ nor $y_{2}(x)$ are eigenfunctions since they do not satisfy our \textit{mixed-boundary conditions}

\vspace{5mm}

Moving on, we consider the case where $\lambda=0$. Plugging-in $\lambda=0$, our ODE becomes $f''(x)=0$. The astute reader might also notice that we have done the work to derive our solutions

\begin{center}

$h_{1}(x)=x$ $\quad$ and $\quad$ $h_{2}(x)=1$

\end{center}

in the previous question, so we won't repeat said work here. Again, in order to apply our \textit{mixed-boundary conditions}, we will have to calculate some derivatives

\begin{center}

$h_{1}(x)=x$ $\quad$ and $\quad$ $h_{2}(x)=1$

\vspace{2mm}

$h_{1}'(x)=1$ $\quad$ and $\quad$ $h_{2}'(x)=0$

\end{center}

Applying our \textit{mixed-boundary conditions}, we find that

\begin{center}

$h_{1}'(x)=1$ $\quad$ and $\quad$ $h_{2}'(x)=0$

\vspace{2mm}

$h_{1}'(0)=1$ $\quad$ and $\quad$ $h_{2}'(0)=0$

\end{center}

Therefore, $h_{2}(x)$ is the only solution that satisfies our \textit{mixed-boundary conditions}

\vspace{5mm}

Finally, we want to consider the case where $\lambda > 0$. Again, using our work from the previous question, we know that our solutions are of the form

\begin{center}

$z_{1}(x)=\cos\bigg(\sqrt{\lambda} x\bigg)$ $\quad$ and $\quad$ $z_{2}(x)=\sin\bigg(\sqrt{\lambda} x\bigg)$

\end{center}

Taking some derivatives, since that allows us to apply our \textit{mixed-boundary conditions}, we find that

\begin{center}

$z_{1}(x)=\cos\bigg(\sqrt{\lambda} x\bigg)$ $\quad$ and $\quad$ $z_{2}(x)=\sin\bigg(\sqrt{\lambda} x\bigg)$

\vspace{2mm}

$z_{1}'(x)=-\sqrt{\lambda} \sin\bigg(\sqrt{\lambda} x\bigg)$ $\quad$ and $\quad$ $z_{2}'(x)=\sqrt{\lambda} \cos\bigg(\sqrt{\lambda} x\bigg)$

\end{center}

Applying our \textit{mixed-boundary conditions} we have

\begin{center}

$z_{1}'(x)=-\sqrt{\lambda} \sin\bigg(\sqrt{\lambda} x\bigg)$ $\quad$ and $\quad$ $z_{2}'(x)=\sqrt{\lambda} \cos\bigg(\sqrt{\lambda} x\bigg)$

\vspace{2mm}

$z_{1}'(0)=-\sqrt{\lambda} \sin\bigg(\sqrt{\lambda} (0)\bigg)$ $\quad$ and $\quad$ $z_{2}'(0)=\sqrt{\lambda} \cos\bigg(\sqrt{\lambda} (0)\bigg)$

\vspace{2mm}

$z_{1}'(0)=-\sqrt{\lambda} \sin(0)$ $\quad$ and $\quad$ $z_{2}'(0)=\sqrt{\lambda} \cos(0)$

\vspace{2mm}

$0=0$ $\quad$ and $\quad$ $0 \neq 1$

\end{center}

Since $z_{1}(x)=\cos\bigg(\sqrt{\lambda} x\bigg)$ satisfies our first \textit{mixed-boundary condition}, we want to make it so that $z_{1}'(L)=0$

\begin{center}

$z_{1}'(x)=-\sqrt{\lambda} \sin\bigg(\sqrt{\lambda} x\bigg)$

\vspace{2mm}

$z_{1}'(L)=-\sqrt{\lambda} \sin\bigg(\sqrt{\lambda} L\bigg)$

\vspace{2mm}

$0=-\sqrt{\lambda} \sin\bigg(\sqrt{\lambda} L\bigg)$

\vspace{2mm}

$0=\sin\bigg(\sqrt{\lambda} L\bigg)$

\vspace{2mm}

$\sin^{-1}(0)=\sin^{-1}\Bigg(\sin\bigg(\sqrt{\lambda} L\bigg)\Bigg)$

\end{center}

So, $(\lambda L)=(2n-1)\frac{\pi}{2}$, where $2n-1$ are odd numbers. So, $\lambda=\bigg(\frac{(n \pi - \frac{\pi}{2})}{L}\bigg)^2,  ~ \forall ~ n \in \mathbb{N}$

Thus, our \textit{ND-Laplacian} eigenfunctions are

\begin{center}

$f_{n,DN}(x)=\cos\bigg(\frac{(n \pi - \frac{\pi}{2})}{L} x\bigg), ~ \forall ~ n \in \mathbb{N}$

\end{center}

\end{proof}

\end{flushleft}

\begin{flushleft}

4) Let the differentiable equation $f''(x)+\lambda f(x)=0$ be given and let $g(x)$ be a twice differentiable function. \\ 

\begin{proof}

Now, multiply $f''(x)+\lambda f(x)=0$ by $g(x)$ to obtain

\begin{center}

$f''(x)g(x)+\lambda f(x)g(x)=0$

\end{center}

Now, we want to integrate $f''(x)g(x)+\lambda f(x)g(x)=0$ by parts. But, before we do, let use the \textit{properties of integrals} to obtain

\begin{center}

$\int_{0}^{L} f''(x)g(x) \ dx + \int_{0}^{L} \lambda f(x)g(x) \ dx=0$

\end{center}

Next, integrate $\int_{0}^{L} f''(x)g(x) \ dx$ by parts

\begin{center}

$u=g(x)$ $\quad$ $du=g''(x)$ $\quad$ $dv=f''(x)$ $\quad$ $v=\int f''(x)=f'(x)+c$

\vspace{2mm}

$\int_{0}^{L} f''(x)g(x)= \bigg[g(x)f'(x)\bigg] \bigg|_{0}^{L} - \int_{0}^{L} f'(x)g'(x) \ dx$

\end{center}

Substituting in for $\int_{0}^{L} f''(x)g(x) \ dx$ with the above result, we have

\begin{center}

$\int_{0}^{L} f''(x)g(x) \ dx=\bigg[g(x)f'(x)\bigg] \bigg|_{0}^{L} - \int_{0}^{L} f'(x)g'(x) \ dx$

\end{center}

We can now re-write $\int_{0}^{L} f''(x)g(x) \ dx + \int_{0}^{L} \lambda f(x)g(x) \ dx=0$ as

\begin{center}

$\bigg[g(x)f'(x)\bigg] \bigg|_{0}^{L} - \int_{0}^{L} f'(x)g'(x) \ dx + \int_{0}^{L} \lambda f(x)g(x) \ dx=0$

\end{center}

To continue, we want to integrate by parts yet again

\begin{center}

$u=g'(x)$ $\quad$ $du=g''(x)$ $\quad$ $dv=f'(x)$ $\quad$ $v=\int f'(x) \ dx= f(x)$

\end{center}

We can now re-write  $\int_{0}^{L} f'(x)g'(x) \ dx$ as

\begin{center}

$\bigg[g'(x)f(x)\bigg]_{0}^{L} - \int_{0}^{L} f(x)g''(x) \ dx$

\end{center}

Substituting back-in, we can re-write $\bigg[g(x)f'(x)\bigg] \bigg|_{0}^{L} - \int_{0}^{L} f'(x)g'(x) \ dx + \int_{0}^{L} \lambda f(x)g(x) \ dx=0$ as

\begin{center}

$\bigg[g(x)f'(x)\bigg]_{0}^{L}\bigg[g'(x)f(x)\bigg]_{0}^{L} - \int_{0}^{L} f(x)g''(x) \ dx+ \int_{0}^{L} \lambda f(x)g(x) \ dx=0$

\end{center}

which is the \textit{Lagrange-Identity} from the beginning of the semester

\vspace{5mm}

Now, we turn our attention to showing the orthogonality of the mixed \textit{DN-Laplacian} eigenfunctions. First, let the differentiable equation $f''(x)+\lambda f(x)=0$ be given and let $g(x)$ be a twice differentiable function. Now, multiply $f''(x)+\lambda f(x)$ to obtain

\begin{center}

$f''(x)g(x)+\lambda f(x)g(x)$

\end{center}

Now, we want to integrate $f''(x)g(x)$ by parts

\begin{center}

$u=g(x) \quad du=g'(x) \quad dv=f''(x) \quad v=\int f''(x)=f'(x)$

\end{center}

We can now re-write $f''(x)g(x)$ as

\begin{center}

$f''(x)g(x)=\bigg[g(x)f'(x)\bigg]_{0}^{L} + \int_{0}^{L} f'(x)g'(x) \ dx$

\end{center}

To continue, we want to integrate by parts yet again

\begin{center}

$u=g'(x) \quad du=g''(x) \quad dv=f'(x) \quad v=\int f'(x)=f(x)$

\end{center}

We can now re-write $\bigg[g(x)f'(x)\bigg]_{0}^{L} + \int_{0}^{L} f'(x)g'(x) \ dx$ as

\begin{center}

$\bigg[g(x)f'(x)\bigg]_{0}^{L} - \bigg[g'(x)f(x)\bigg]_{0}^{L} - \int_{0}^{L} f(x)g''(x) \ dx$

\end{center}

So, at this point, we have

\begin{center}

$f''(x)g(x)=\bigg[g(x)f'(x)\bigg]_{0}^{L} - \bigg[g'(x)f(x)\bigg]_{0}^{L} - \int_{0}^{L} f(x)g''(x) \ dx$

\end{center}

Now, if we make the substitution $g(x)=f_{m,DN}(x)$ and $f(x)=f_{n,DN}(x)$, we have

\begin{center}

$\int_{0}^{L} f''_{n, DN}(x)f_{m, DN}(x) \ dx=\bigg[f_{m, DN}(x)f'_{n, DN}(x)\bigg]_{0}^{L} - \bigg[f'{m, DN}(x)f_{n, DN}(x)\bigg]_{0}^{L} - \int_{0}^{L} f_{n, DN}(x)f''_{m, DN}(x) \ dx$

\end{center}

Looking at the $\bigg[f_{m, DN}(x)f'_{n, DN}(x)\bigg]_{0}^{L}$ term, if we apply the boundary conditions

\begin{center}

\[ \begin{cases} 
      f(0)=0 \\
      f'(L)=0
   \end{cases}
\]

\end{center}

then this term will go to zero. This leaves us with

\begin{center}

$\int_{0}^{L} f''_{n, DN}(x)f_{m, DN}(x) \ dx=- \bigg[f'{m, DN}(x)f_{n, DN}(x)\bigg]_{0}^{L} - \int_{0}^{L} f_{n, DN}(x)f''_{m, DN}(x) \ dx$

\end{center}

Then, if we look at the $- \bigg[f'{m, DN}(x)f_{n, DN}(x)\bigg]_{0}^{L}$ term, we can see that if we again apply the boundary conditions

\begin{center}

\[ \begin{cases} 
      f(0)=0 \\
      f'(L)=0
   \end{cases}
\]

\end{center}

this term will also go to zero. So, are we are left with is

\begin{center}

$\int_{0}^{L} f''_{n, DN}(x)f_{m, DN}(x) \ dx=- \int_{0}^{L} f_{n, DN}(x)f''_{m, DN}(x) \ dx$

\end{center}

Now, if we return to our twice differentiable function

\begin{center}

$f''(x)+\lambda f(x)=0$

\end{center}

and re-arrange terms, we have

\begin{center}

$f''(x)=-\lambda f(x)$

\end{center}

If we substitute this into our integral, we have

\begin{center}

$\int_{0}^{L} -\lambda f_{n, DN}(x)f_{m, DN}(x)=-\int_{0}^{L} f_{n, DN}(x) \cdot -\lambda f_{m, DN}(x) \ dx$

\end{center}

Simplifying, we have

\begin{center}

$-\int_{0}^{L} f_{n, DN}(x) \cdot -\lambda f_{m, DN}(x) \ dx=\int_{0}^{L} -\lambda f_{n, DN}(x)f_{m, DN}(x) \ dx$

\vspace{2mm}

$\implies -(-\lambda) \int_{0}^{L} f_{n, DN}(x)f_{m, DN}(x) \ dx=-\lambda \int_{0}^{L} f_{n, DN}(x)f_{m, DN}(x) \ dx$

\vspace{2mm}

$\implies -(-\lambda_{n}) \int_{0}^{L} f_{n, DN}(x)f_{m, DN}(x) \ dx + \lambda_{m} \lambda \int_{0}^{L} f_{n, DN}(x)f_{m, DN}(x) \ dx$

\vspace{2mm}

$\implies -(-\lambda_{n}) \int_{0}^{L} \lambda_{m} f_{n, DN}(x)f_{m, DN}(x) \ dx=0$

\vspace{2mm}

$\implies -(\lambda_{m}-\lambda_{n}) \int_{0}^{L} \lambda_{m} f_{n, DN}(x)f_{m, DN}(x) \ dx=0$

\end{center}

Now, since $m$ and $n$ are assumed to be different, we know that $(\lambda_{m}-\lambda_{n})$ cannot be equal to zero. Thus, $\int_{0}^{L} \lambda_{m} f_{n, DN}(x)f_{m, DN}(x) \ dx$ has to be equal to zero, as desired.
Hence, we have shown the orthogonality of the mixed \textit{DN-Laplacian} eigenfunctions.

\vspace{5mm}

We now move on to show the orthogonality of the mixed \textit{ND-Laplacian} eigenfunctions. Like the calculation we just finished, let the differentiable equation $f''(x)+\lambda f(x)=0$ be given and let $g(x)$ be a twice differentiable function. Now, multiply $f''(x)+\lambda f(x)$ to obtain

\begin{center}

$f''(x)g(x)+\lambda f(x)g(x)$

\end{center}

Now, we want to integrate $f''(x)g(x)$ by parts

\begin{center}

$u=g(x) \quad du=g'(x) \quad dv=f''(x) \quad v=\int f''(x)=f'(x)$

\end{center}

We can now re-write $f''(x)g(x)$ as

\begin{center}

$f''(x)g(x)=\bigg[g(x)f'(x)\bigg]_{0}^{L} + \int_{0}^{L} f'(x)g'(x) \ dx$

\end{center}

To continue, we want to integrate by parts yet again

\begin{center}

$u=g'(x) \quad du=g''(x) \quad dv=f'(x) \quad v=\int f'(x)=f(x)$

\end{center}

We can now re-write $\bigg[g(x)f'(x)\bigg]_{0}^{L} + \int_{0}^{L} f'(x)g'(x) \ dx$ as

\begin{center}

$\bigg[g(x)f'(x)\bigg]_{0}^{L} - \bigg[g'(x)f(x)\bigg]_{0}^{L} - \int_{0}^{L} f(x)g''(x) \ dx$

\end{center}

So, at this point, we have

\begin{center}

$f''(x)g(x)=\bigg[g(x)f'(x)\bigg]_{0}^{L} - \bigg[g'(x)f(x)\bigg]_{0}^{L} - \int_{0}^{L} f(x)g''(x) \ dx$

\end{center}

Now, if we make the substitution $g(x)=f_{m,ND}(x)$ and $f(x)=f_{n,ND}(x)$, we have

\begin{center}

$\int_{0}^{L} f''_{n, ND}(x)f_{m, ND}(x) \ dx=\bigg[f_{m, ND}(x)f'_{n, ND}(x)\bigg]_{0}^{L} - \bigg[f'{m, ND}(x)f_{n, ND}(x)\bigg]_{0}^{L} - \int_{0}^{L} f_{n, DN}(x)f''_{m, ND}(x) \ dx$

\end{center}

Looking at the $\bigg[f_{m, ND}(x)f'_{n, ND}(x)\bigg]_{0}^{L}$ term, if we apply the boundary conditions

\begin{center}

\[ \begin{cases} 
      f(0)=0 \\
      f'(L)=0
   \end{cases}
\]

\end{center}

then this term will go to zero. This leaves us with

\begin{center}

$\int_{0}^{L} f''_{n, ND}(x)f_{m, ND}(x) \ dx=- \bigg[f'{m, ND}(x)f_{n, ND}(x)\bigg]_{0}^{L} - \int_{0}^{L} f_{n, ND}(x)f''_{m, ND}(x) \ dx$

\end{center}

Then, if we look at the $- \bigg[f'{m, ND}(x)f_{n, ND}(x)\bigg]_{0}^{L}$ term, we can see that if we again apply the boundary conditions

\begin{center}

\[ \begin{cases} 
      f(0)=0 \\
      f'(L)=0
   \end{cases}
\]

\end{center}

this term will also go to zero. So, are we are left with is

\begin{center}

$\int_{0}^{L} f''_{n, ND}(x)f_{m, ND}(x) \ dx=- \int_{0}^{L} f_{n, ND}(x)f''_{m, ND}(x) \ dx$

\end{center}

Now, if we return to our twice differentiable function

\begin{center}

$f''(x)+\lambda f(x)=0$

\end{center}

and re-arrange terms, we have

\begin{center}

$f''(x)=-\lambda f(x)$

\end{center}

If we substitute this into our integral, we have

\begin{center}

$\int_{0}^{L} -\lambda f_{n, ND}(x)f_{m, ND}(x)=-\int_{0}^{L} f_{n, ND}(x) \cdot -\lambda f_{m, ND}(x) \ dx$

\end{center}

Simplifying, we have

\begin{center}

$-\int_{0}^{L} f_{n, ND}(x) \cdot -\lambda f_{m, ND}(x) \ dx=\int_{0}^{L} -\lambda f_{n, DN}(x)f_{m, ND}(x) \ dx$

\vspace{2mm}

$\implies -(-\lambda) \int_{0}^{L} f_{n, ND}(x)f_{m, ND}(x) \ dx=-\lambda \int_{0}^{L} f_{n, ND}(x)f_{m, ND}(x) \ dx$

\vspace{2mm}

$\implies -(-\lambda_{n}) \int_{0}^{L} f_{n, ND}(x)f_{m, ND}(x) \ dx + \lambda_{m} \lambda \int_{0}^{L} f_{n, ND}(x)f_{m, ND}(x) \ dx$

\vspace{2mm}

$\implies -(-\lambda_{n}) \int_{0}^{L} \lambda_{m} f_{n, ND}(x)f_{m, ND}(x) \ dx=0$

\vspace{2mm}

$\implies -(\lambda_{m}-\lambda_{n}) \int_{0}^{L} \lambda_{m} f_{n, ND}(x)f_{m, ND}(x) \ dx=0$

\end{center}

Now, since $m$ and $n$ are assumed to be different, we know that $(\lambda_{m}-\lambda_{n})$ cannot be equal to zero. Thus, $\int_{0}^{L} \lambda_{m} f_{n, DN}(x)f_{m, DN}(x) \ dx$ has to be equal to zero, as desired.
Thus, we have shown the orthogonality of the mixed \textit{ND-Laplacian} eigenfunctions.

\end{proof}

5) Looking at the \textit{eigendata} from these two mixed problems, we notice that they are similar in that they both are of the form

\begin{center}

$\lambda=\bigg(\frac{(n \pi - \frac{\pi}{2})}{L}\bigg)^2,  ~ \forall ~ n \in \mathbb{N}$

\end{center}

yet they differ in the form of their associated \textit{eigenfunctions}.

\end{flushleft}

\begin{flushleft}

6)Looking at the \textit{eigendata} from our mixed \textit{DN-Laplacian} eigenproblem and our regular \textit{Dirichlet} eigenproblem, we see that for the \textit{DN-Laplacian} eigenproblem our $n$ values in the $\lambda$ expression are only odd-numbers $\in \mathbb{N}$, while for the \textit{Dirichlet} eigenproblem, we considered all values of $n \in \mathbb{N}$.

\end{flushleft}

\end{document}