\documentclass[executivepaper]{article}

\usepackage{mathtools}

\usepackage{amssymb}

\usepackage{kantlipsum,graphicx}

\usepackage{amsmath}

\usepackage{commath}

\usepackage{amsthm}

\usepackage[utf8]{inputenc}

\everymath{\displaystyle}

\begin{document}

\vspace*{-40mm}

\begin{center}

Homework 4

\end{center}

\begin{flushright}

Brendan Busey

\end{flushright}

\begin{flushleft}

1)

\begin{proof}

Let $w(x,t)=u(-x,t)-u(x,t)$. Then, $w(x,t)$ satisfies the differential equation and, 

\begin{center}

$w(x,0)=u(-x,0)-u(x,0)=\varphi(-x)-\varphi(x)=\varphi(x)-\varphi(x)=0$

\end{center}

Thus, we have the initial condition diffusion equation

\begin{center}

$w_{t}=kW_{xx}$
$w(x,0)=0$

\end{center}

By uniqueness, $w(x,0) \equiv 0$. Then,

\begin{center}

$u(-x,t)-u(x,t)=0$

$u(-x,t)=u(x,t)$

\end{center}

Thus, $u(x,t)$ is an even function of $x$. Since $u(x,t)$ is an even function, we know that $u(x,t)$ will be symmetric around the graph representing the physical situation, meaning that there will be a situation where two different points in that domain will give you the same diffusion level and that at other points, the diffusion level will be zero.

\end{proof}

\end{flushleft}

\begin{flushleft}

2) Consider the $Q$-problem

\begin{center}

$\frac{\partial Q}{\partial t}=D \frac{\partial^2 Q}{\partial x^2}$ {~} for {~} $-\infty < x < \infty$,  $t > 0$

\vspace{3mm}

\[
 Q(x,0)=
  \begin{cases} 
      \hfill 0    \hfill & \text{ if $x <$ 0} \\[2ex]
      \hfill -1 \hfill & \text{ if $x >$ 0} \\
  \end{cases}
\]

\vspace{3mm}

Now, suppose $Q(x,t)=g \bigg(\frac{x}{\sqrt{4Dt}}\bigg)$, for some one-variable function $g$.

\vspace{2mm}

Next, let $\xi=\frac{x}{\sqrt{4Dt}}$. Then, $\frac{\partial Q}{\partial t}=g'(\xi)\bigg(-\frac{1}{2}x(4Dt)^{-\frac{3}{2}}(4D)\bigg)$.

\vspace{2mm}

Looking more closely at $g'(\xi)$, we have

\vspace{2mm}

$g'(\xi)=\bigg(-\frac{x4D}{2(4Dt) \sqrt{4Dt}}\bigg)$

\vspace{2mm}

Returning to $\frac{\partial Q}{\partial t}$ and simplifying, we have

\pagebreak

\vspace*{-40mm}

$\frac{\partial Q}{\partial t}=g'(\xi)\bigg(-\frac{x}{2t \sqrt{4Dt}}\bigg)$

\vspace{2mm}

$\frac{\partial Q}{\partial t}=g'(\xi)\bigg(-\frac{1}{2t} \xi \bigg)$

\vspace{5mm}

Now, if we look at derivatives with respect with to $x$, we get

\vspace{3mm}

$\frac{\partial Q}{\partial x}=g'(\xi)\bigg(\frac{1}{\sqrt{4Dt}}\bigg) \quad$ and $\quad \frac{\partial^2 Q}{\partial x^2}=g''(\xi)\bigg(\frac{1}{\sqrt{4Dt}}\bigg)$

\vspace{3mm}

Then, if we substitute the derivatives we just calculated into the PDE, we see it transform into

\vspace{2mm}

$-\frac{1}{2t} \xi g'(\xi)=D \frac{1}{4Dt} g''(\xi)$

\vspace{1mm}

If we then multiply through by $4t$, we get

\vspace{1mm}

$-\frac{4}{2} \xi g'(\xi)=g''(\xi)$

\vspace{1mm}

$g''(\xi)+2\xi g'(\xi)=0$

\vspace{3mm}

Now, let $w(\xi)=g'(\xi) \quad$ and $\quad w'(\xi)=g''(\xi)$. Substituting these into our differential equation, we get

\vspace{2mm}

$w'(\xi)+2\xi w(\xi)=0$

So, if we remember our ``training", in order to solve this differential equation, the first thing we need to do is calculate the integrating factor

\vspace{2mm}

$e^{\int_{0}^{\xi} 2s \ ds}=e^{\xi^{2}}$

\vspace{2mm}

Then, if we multiply through by the integrating factor, we have

\vspace{2mm}

$e^{\xi^{2}} w'(\xi)+2\xi w(\xi)=0$

\vspace{2mm}

$\frac{d}{d\xi} \bigg[e^{\xi^{2}} w(\xi)\bigg]=0$

\vspace{2mm}

$\int_{0}^{\xi} \frac{d}{ds} \bigg[e^{s^{2}} w(s)\bigg]=0$

\vspace{2mm}

Continuing on, by the \textit{Fundamental Theorem of Calculus}, we have

\vspace{2mm}

$\bigg[e^{s^{2}} w(s)\bigg] \Big |_{s=0}^{s=\xi}$

\vspace{2mm}

Which, simplifies to the following if we evaluate it

\vspace{2mm}

$e^{\xi^{2}}w(\xi)-w(0)=0$

\vspace{2mm}

$w(\xi)=w(0)e^{\xi^2}$

\vspace{2mm}

Now, if we back substitute for $w$, we get

\vspace{2mm}

$g'(\xi)=g'(0)e^{\xi^2}$

\vspace{2mm}

$\int_{0}^{\xi} g'(s) ds=g'(0) \int_{0}^{\xi} e^{-s^{2}} ds$

\vspace{2mm}

$g(\xi)=g(0)+g'(0) \int_{0}^{\xi} e^{-s^{2}}$

\pagebreak

\vspace*{-40mm}

Now, if we recall 

\[
 Q(x,0)=
  \begin{cases} 
      \hfill 0    \hfill & \text{ if $x <$ 0} \\[2ex]
      \hfill -1 \hfill & \text{ if $x >$ 0} \\
  \end{cases}
\]

\vspace{2mm}

Then, for $x > 0$, we have 

\vspace{2mm}

$1=Q(x,0)=\lim_{x \to 0^{+}} Q(x,t)=g(0)+g'(0) \int_{0}^{+\infty} e^{-s^{2}} ds$

\vspace{2mm}

$1=g(0)+g'(0) \int_{0}^{+\infty} e^{-s^{2}} ds=g(0)+g'(0) \frac{\sqrt{\pi}}{2}$

\vspace{2mm}

$1=g(0)+g'(0) \frac{\sqrt{\pi}}{2}$

\vspace{5mm}

If we then look at the case where $x < 0$, we have

\vspace{2mm}

$0=Q(x,0)=\lim_{x \to 0^{-}} Q(x,t)=g(0)+g'(0) \int_{0}^{+\infty} e^{-s^{2}} ds$

$0=g(0)+g'(0) \int_{0}^{+\infty} e^{-s^{2}} ds=g(0)+g'(0) \frac{\sqrt{\pi}}{2}$

\vspace{2mm}

$1=g(0)+g'(0) \frac{\sqrt{\pi}}{2}$

\vspace{5mm}

Finally, if we add our results for the above two cases we considered, we have

\vspace{2mm}

$g(0)=\frac{1}{2}$

\vspace{1mm}

$g'(0)=\frac{1}{\sqrt{\pi}}$

\vspace{2mm}

$Q(x,t)=g(\xi)=\frac{1}{2}+\frac{1}{\pi} \int_{0}^{\xi} e^{-s^{2}} ds$

\vspace{2mm}

$Q(x,t)=g(\xi)=\frac{1}{2}+\frac{1}{\pi} \int_{0}^{\frac{x}{\sqrt{4Dt}}} e^{-s^{2}} ds$

\vspace{2mm}

$Q(x,t)=g(\xi)=\frac{1}{2}+\frac{1}{2} {erf} \bigg(\frac{x}{\sqrt{4Dt}}\bigg)$

\end{center}

\vspace {5mm}

Now, if we consider

\begin{center}

\[
 \varphi(x)=
  \begin{cases} 
      \hfill 0    \hfill & \text{ if $x <$ -L} \\[2ex]
      \hfill \sqrt{\pi} \hfill & \text{ if $-L \leq x \leq L$} \\[2ex]
      \hfill 0     \hfill & \text{if $x > L$}
  \end{cases}
\]

\pagebreak

\vspace*{-40mm}

we can note a few things. The first being that whenever $x$ is less than $-L$, then we have that $\varphi$ is zero which indicates that the diffusion of kool-aid through the tube is also zero. Next, we note that when 
$-L \leq x \leq L$, we have that $\varphi$ is equal to $\sqrt{\pi}$, so we know that the diffusion throughout the tube in this case is linear. Finally, whenever $x$ is greater than $L$, we have that $\varphi$ is equal to zero which indicates that the diffusion of kool-aid through the tube is, again, zero.

\end{center}

\end{flushleft}

\begin{flushleft}

3) Suppose that if $u_{1}(x,t)$ and $u_{2}(x,t)$ are solutions to the given problem, then $u_{1}(x,t)=u_{2}(x,t) {~} \forall {~} (x,t) \in {~} \Omega_{L,\infty}$, suppose

\begin{proof}

To begin, define $w(x,t)=u_{1}(x,t)-u_{2}(x,t)$. 

\vspace{2mm}

At this point, we know that $w$ satisfies the PDE $\frac{\partial w}{\partial t}-D\frac{\partial^2}{\partial x^2}=0$ with the initial condition of $w(x,0)=0$ and the boundary conditions $w(0,t)=0$ and $w(L,t)=0$. Okay, now, multiply $\frac{\partial w}{\partial t}-D\frac{\partial^2}{\partial x^2}=0$ by $w$ to obtain

\begin{center}

$w\frac{\partial w}{\partial t}-Dw\frac{\partial^2 w}{\partial x^2}=0$

\end{center}

Now, before continuing, let us note the following:

\begin{center}

1) $\frac{\partial}{\partial t}\bigg[\frac{1}{2}w^2\bigg]=w\bigg(\frac{\partial w}{\partial t}\bigg)$

\vspace{2mm}

2) $\frac{\partial}{\partial x}\bigg[-Dw\frac{\partial w}{\partial x}\bigg]=-Dw\frac{\partial^2 w}{\partial x^2}-D\bigg(\frac{\partial w}{\partial x}\bigg)^2$

\vspace{2mm}

3) $\frac{\partial}{\partial x}\bigg[-Dw\frac{\partial w}{\partial x}\bigg]+D\bigg(\frac{\partial w}{\partial x}\bigg)^2=-Dw\frac{\partial^2 w}{\partial x^2}$

\end{center}

Using the above points we just noted, $w\frac{\partial w}{\partial t}-Dw\frac{\partial^2 w}{\partial x^2}=0$ becomes

\begin{center}

$\frac{\partial}{\partial t}\bigg[\frac{1}{2} w^2\bigg]+\frac{\partial}{\partial x}\bigg[-Dw\bigg(\frac{\partial w}{\partial x}\bigg)\bigg]+D\bigg(\frac{\partial w}{\partial x}\bigg)^2=0$

\end{center}

\vspace{2mm}

Next, we want to integrate over the interval $0 \leq x \leq L$, giving us

\begin{center}

$\int_{0}^{L} \frac{\partial}{\partial t} \bigg[\frac{1}{2} w^2\bigg] dx + \int_{0}^{L} \frac{\partial}{\partial x} \bigg[-Dw \bigg(\frac{\partial w}{\partial x}\bigg)\bigg] + D \int_{0}^{L} \bigg(\frac{\partial w}{\partial x}\bigg)^2 dx=0$

\end{center}

which simplifies to, with help from the \textit{Fundamental Theorem of Calculus},

\begin{center}

$\frac{d}{dt} \int_{0}^{L} \frac{1}{2}w^2 dx + \bigg[-Dw \bigg(\frac{\partial w}{\partial x}\bigg)\bigg] \Big |_{x=0}^{x=L} + D \int_{0}^{L} \bigg(\frac{\partial w}{\partial x}\bigg)^2 dx=0$

\end{center}

\vspace{2mm}

Next, if we use the provided boundary conditions, the $\bigg[-Dw \bigg(\frac{\partial w}{\partial x}\bigg)\bigg] \Big |_{x=0}^{x=L}$ vanishes since, according to our boundary conditions

\vspace{2mm}

when we ``plug-in" both $x$ and $L$ into our $w$ function, we get zero each time. So, after doing some rearrangement, we come up with the following

\begin{center}

$\frac{d}{dt} \int_{0}^{L} \frac{1}{2}w^2 dx=-D \int_{0}^{L} \bigg(\frac{\partial w}{\partial x}\bigg)^2 dx \leq 0$

\end{center}

\pagebreak

\vspace*{-40mm}

Thus, we have that the function $F(t)=\int_{0}^{L} \frac{1}{2}w(x,t)^2 dx$, by the above function we just derived, is a decreasing function. So, since $F(0) \geq F(t)$ for any $t > 0$, we know that $\int_{0}^{L} w(x,0)^2 dx \geq \int_{0}^{L} w(x,t)^2 dx$. So, $0 \geq \int_{0}^{L} w(x,t)^2 dx {~} \forall {~} t > 0$. The previous statements imply that $w(x,t)=0 {~} \forall {~} (x,t) \in {~} \Omega_{L,\infty}$, which in turn implies that $u_{1}(x,t)=u_{2}(x,t) \in {~} \Omega_{L,\infty}$.

\end{proof}

\end{flushleft}

\end{document}