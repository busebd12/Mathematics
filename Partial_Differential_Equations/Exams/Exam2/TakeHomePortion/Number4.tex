\documentclass[12pt, executivepaper]{article}
\usepackage{mathtools}
\usepackage{outlines}
\usepackage{amsfonts}
\usepackage{booktabs}
\usepackage{tikz}
\usepackage{commath}
\usepackage{amsthm}
\usepackage{amsmath}
\everymath{\displaystyle}

\begin{document}

\vspace*{-40mm}

\begin{flushleft}

4) Let the \textit{Diffusion Equation} $\frac{\partial u}{\partial t}=D\frac{\partial^2 u}{\partial x^2}$ be given as defined. \\

\begin{proof}

First, suppose that solutions are of the form $u=u(x,t)$. After moving things around, we have

\begin{center}

$\frac{\partial u}{\partial t}-D\frac{\partial^2 u}{\partial x^2}=0$

\end{center} 

Now, multiply $\frac{\partial u}{\partial t}-D\frac{\partial^2 u}{\partial x^2}=0$ through by $u$ to get

\begin{center}

$u \frac{\partial u}{\partial t}-Du \frac{\partial^2 u}{\partial x^2}=0$

\end{center}

Now, before continuing, we note the following

\vspace{2mm}

\begin{center}

$\frac{\partial}{\partial t} \bigg[\frac{1}{2} u^2\bigg]=u \bigg(\frac{\partial u}{\partial t}\bigg)$

\end{center}

Also,

\begin{center}

$\frac{\partial}{\partial x} \bigg[-Du \bigg(\frac{\partial u}{\partial x}\bigg) \bigg]=-Du \frac{\partial^2 u}{\partial x^2} -D \bigg(\frac{\partial u}{\partial x}\bigg)^2$

\vspace{2mm}

$\frac{\partial}{\partial x} \bigg[-Du \bigg(\frac{\partial u}{\partial x}\bigg) \bigg]+D \bigg(\frac{\partial u}{\partial x}\bigg)^2=-Du \frac{\partial^2 u}{\partial x^2}$

\end{center}

Okay, now, $u \frac{\partial u}{\partial t}-Du \frac{\partial^2 u}{\partial x^2}=0$ becomes

\begin{center}

$\frac{\partial}{\partial t} \bigg[\frac{1}{2} u^2\bigg] + \frac{\partial}{\partial x} \bigg[-Du \bigg(\frac{\partial u}{\partial x}\bigg) \bigg] + D \bigg(\frac{\partial u}{\partial x}\bigg)^2=0$

\end{center}

Then, we integrate over the interval $0 < x < L$ to get

\begin{center}

$\int_{0}^{L} \frac{\partial}{\partial t} \bigg[\frac{1}{2} u^2\bigg] \ dx + \int_{0}^{L} \frac{\partial}{\partial x} \bigg[-Du \bigg(\frac{\partial u}{\partial x}\bigg)\bigg] \ dx + D \int_{0}^{L} \bigg(\frac{\partial u}{\partial x}\bigg)^2 \ dx=0$

\end{center}

After using the \textit{Fundamental Theorem of Calculus}, we have

\begin{center}

$\frac{d}{dt}\int_{0}^{L} \frac{1}{2} u^2 \ dx + \bigg[-Du \bigg(\frac{\partial u}{\partial x}\bigg)\bigg] \Big|_{x=0}^{x=L}+ D \int_{0}^{L} \bigg(\frac{\partial u}{\partial x}\bigg)^2 \ dx=0$

\end{center}

Now, if we evaluate the middle term from $x=0$ to $x=L$, we can see that it becomes

\begin{center}

$-Du(L,t) \frac{\partial u}{\partial x}(L,t) + Du(0,t) \frac{\partial u}{\partial x}(0,t)$

\end{center}

\pagebreak

\vspace*{-40mm}

So, now we have the following

\begin{center}

$\frac{d}{dt}\int_{0}^{L} \frac{1}{2} u^2 \ dx + \bigg(-Du(L,t) \frac{\partial u}{\partial x}(L,t) + Du(0,t) \frac{\partial u}{\partial x}(0,t)\bigg)+ D \int_{0}^{L} \bigg(\frac{\partial u}{\partial x}\bigg)^2 \ dx=0$

\end{center}

Next, remembering that our boundary conditions are

\begin{center}

$-\frac{\partial u}{\partial x}(0,t) + b_{0}u(0,t)=0 \quad$ and $\quad \frac{\partial u}{\partial x}(L,t) + b_{L} u(L,t)$

\end{center}

if we rearrange things a bit, we get

\begin{center}

$-\frac{\partial u}{\partial x}(0,t)=-b_{0}u(0,t) \quad$ and $\quad \frac{\partial u}{\partial x}(L,t)=-b_{L} u(L,t)$

\end{center}

So, upon comparing $-Du(L,t) \frac{\partial u}{\partial x}(L,t) + Du(0,t) \frac{\partial u}{\partial x}(0,t)$ with 

\begin{center}

$-\frac{\partial u}{\partial x}(0,t)=-b_{0}u(0,t)$ and $\frac{\partial u}{\partial x}(L,t)=-b_{L} u(L,t)$

\end{center}

we can substitute using the boundary conditions giving us

\begin{center}

$Du(L,t) b_{L} u(L,t) + Du(0,t) b_{0}u(0,t)$

\end{center}

So, at this point, we have

\begin{center}

$\frac{d}{dt}\int_{0}^{L} \frac{1}{2} u^2 \ dx + \bigg(Du(L,t) b_{L} u(L,t) + Du(0,t) b_{0}u(0,t)\bigg)+ D \int_{0}^{L} \bigg(\frac{\partial u}{\partial x}\bigg)^2 \ dx=0$

\end{center}

Okay, now, if we note that both the second and third term have a $D$ term that we can factor out, after combing terms, we have

\begin{center}

$\frac{d}{dt}\int_{0}^{L} \frac{1}{2} u^2 \ dx + D\bigg[u(L,t) b_{L} u(L,t) + u(0,t) b_{0}u(0,t) + \int_{0}^{L} \bigg(\frac{\partial u}{\partial x}\bigg)^2 \ dx\bigg]=0$

\end{center}

Then, if we move things around, we have

\begin{center}

$\frac{d}{dt}\int_{0}^{L} \frac{1}{2} u^2 \ dx=-D\bigg[u(L,t) b_{L} u(L,t) + u(0,t) b_{0}u(0,t) + \int_{0}^{L} \bigg(\frac{\partial u}{\partial x}\bigg)^2 \ dx\bigg] \leq 0$

\end{center}

\pagebreak

\vspace*{-40mm}

Thus, the function $F(t)=\frac{d}{dt} \int_{0}^{L} \frac{1}{2} u(x,t)^2 \ dx$ by the above expression is a decreasing function.

\end{proof}

\end{flushleft}

\end{document}